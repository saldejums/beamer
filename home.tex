\documentclass[envcountsect]{beamer}

\usetheme{Copenhagen}
\usefonttheme{professionalfonts}
\usepackage{amsthm,amssymb,amsmath}
\usepackage{fontspec}
\usepackage{breqn}

\setbeamertemplate{theorems}[numbered]

\newtheorem*{theorem*}{Theorem}
\newtheorem{prop}{Proposition}

\author{Emils Kalugins}

\title{On the Mason-Stothers theorem}

\institute{University of Latvia}

\date{\today}

\begin{document}
\begin{frame}
\maketitle

\end{frame}

\section{Introduction}

\begin{frame}
\frametitle{Goal}

\begin{theorem*}[Mason-Stothers theorem]
Let $f,g,h$ be non-constant relatively prime polynomials satisfying $f+g=h$
Then \[\operatorname{deg}f, \operatorname{deg}g, \operatorname{deg}h \leq
  n_0(fgh) -1.\]
\end{theorem*}
\end{frame}

\begin{frame}
\frametitle{Justification}
\begin{theorem}
Let $n$ be an integer $\geq 3.$ There is no solution of the equation \[u^n +
  v^n = w^n\] with non-constant relatively prime polynomials $u,v,w$. 
\end{theorem}

\end{frame}

\section{Refresher}
\begin{frame}
\frametitle{Polynomials}
We will work over an algebraically closed field $F$ of characteristic 0, the complex
numbers if you wish.

Let $f(x) \in K[x]$ be a non-zero polynomial, with its factorization
\begin{equation} \label{eq:fac}
  f(x) = c\prod_{i=1}^r(t-\alpha_i)^{m_i}=c(t-\alpha_1)^{m_{1}} \dots (t-\alpha_r)^{m_r},
\end{equation}
with a non-zero constant $c$, and the distinct roots $\alpha_i \; (i=1,\dots,r).$
\end{frame}

\begin{frame}
\frametitle{Polynomials}

It is convenient to write the factorization of $f(x)$ in the form
\[
  f(x) = (x-\alpha)^{m(\alpha)}g(x)
\]
where \( g ( \alpha ) \neq 0 \) and $ m ( \alpha ) $ is the
multiplicity of $\alpha$. If $ \alpha = \alpha_k $ for some index $k$, then
\[
f(x) = c(x- \alpha_k )^{m_k}\prod_{i \neq k}(x-\alpha_i)^{m_i},
\]
and $g(x) = c\prod_{i \neq k}(x-\alpha_i)^{m_i}$.


\end{frame}

\begin{frame}
\section{Preliminaries}
\frametitle{Statement}
\begin{lemma}
Let $f(x)$ be a polynomial over an algebraically closed field. Let $\alpha$ be a
root of $f$ with multiplicity $m(\alpha)$. Then the multiplicity of $\alpha$ in
$f'(x)$ is $m(\alpha) - 1.$ 
\end{lemma}
\end{frame}

\begin{frame}
\frametitle{Proof}
\begin{proof}
Write $f(x) = (x-\alpha)^{m}g(t)$ with $g(\alpha) \neq 0.$ By taking the
derivative of $f$, we get.
\begin{align*}
  f'(x) &= (x-\alpha)^mg'(x) + m(x-\alpha)^{m-1}g(x) \\
        &= (x-\alpha)^{m-1}((x-\alpha)g'(x)+mg(t)) = (x-\alpha)^{m-1}h(x)
\end{align*}  
where $h(x) = ((x-\alpha)g'(x)+mg(x))$. Obviously $h(\alpha) \neq 0.$ So that
$m-1$ is the highest power such that $(x-\alpha)^{m-1}$ divides $f'(x).$

\end{proof}
\end{frame}

\begin{frame}
  \frametitle{Statement}
  Define $n_0(f)$ to be the number of distinct roots of $f$.
  \begin{prop}
   Let $f$ be a non-constant polynomial. Suppose that $f(t)$ has the
   factorization \eqref{eq:fac}. Then the $g.c.d. \; j(f,f')$ is
   \[
     (f,f') = c_1\prod_{i=1}^r(x-\alpha)^{m_i-1},
   \]
   with some constant $c_1.$ In particular,
   \[
     \operatorname{deg}(f,f') = \operatorname{deg}f - n_0(f).
   \]
  \end{prop}
\end{frame}

\section{The Mason-Stothers theorem}
\begin{frame}
\frametitle{Significance}

It is immediate that if $f,g$ are non-zero polynomials, then
\[
n_0(fg) \leq n_0(f) + n_0(g).
\]
If in addition $f,g$ are coprime, then equality holds.

It is obvious that $\operatorname{deg} f$ can be very large, but $n_0(f)$ may be
small.

\begin{example}
  \[
f(x) = (x-\alpha)^{1000}
 \]
 The polynomial $f$ has degree 1000, but $n_0(f) = 1.$
\end{example}
The Mason-Stothers theorem gives a remarkable additive condition under which the degree cannot be large.
\end{frame}

\begin{frame}
\frametitle{Formulation}

\begin{theorem*}[Mason-Stothers theorem]
Let $f,g,h$ be non-constant relatively prime polynomials satisfying $f+g=h$
Then \[\operatorname{deg}f, \operatorname{deg}g, \operatorname{deg}h \leq
  n_0(fgh) -1.\]
\end{theorem*}

\end{frame}

\begin{frame}
  \frametitle{Proof}
  \begin{proof}\let\qed\relax
    We first note the identity
    \[
      f'g - fg' = f'h - fh'.
    \]
    This trivially follows from $f'+g' = h'.$ \\
    We have $f'g - fg' \neq 0,$
    otherwise $g$ would divide $g'$ since $f,g$ are relatively prime and non-constant. \\
    Then we noteice that the g.c.d. $(f, f')$ divides the left side and so does
    $(g,g')$, and $(h,h')$ divides the right side, which is equal to the left
    side. Therefore, since $f,g,h$ are relatively prime,
    $$
      \text{the product } (f,f')(g,g')(h,h') \text{ divides } f'g - fg'.
    $$

  \end{proof}
\end{frame}

\begin{frame}
  \frametitle{Proof}
  \begin{proof}\let\qed\relax
    This yields an inequality between the degress, namely,
    \begin{align} 
      \operatorname{deg}(f,f') + \operatorname{deg}(g,g') + \operatorname{deg}(h,h') &\leq
      \operatorname{deg}(f'g - fg') \label{ineq} \\ 
      &\leq \operatorname{deg}f + \operatorname{deg}g - 1 \nonumber
    \end{align}
    We now use Proposition 1 for each of $f,g,h:$
    \begin{align*}
      \operatorname{deg}(f,f') &= \operatorname{deg}f - n_0(f) \\
      \operatorname{deg}(g,g') &= \operatorname{deg}g - n_0(g) \\
      \operatorname{deg}(h,h') &= \operatorname{deg}h - n_0(h)
    \end{align*}
    \end{proof}


\end{frame}

\begin{frame}
  \frametitle{Proof}
  \begin{proof}
We substitue these equalities in \eqref{ineq}and, after simplifying, get the
following:
\[
 \operatorname{deg}h \leq n_0(f) + n_0(g) + n_0(h) - 1 = n_0(fgh) - 1.
\]
The result for $f,g$ follows from the fact that $f'g - fg' = f'h - fh' =
h'g-hg'$ and by changing the appropriate value in \eqref{ineq}.

  \end{proof}
\end{frame}

\section{Consequences}
\begin{frame}
\frametitle{Fermat's analogue} 
\begin{theorem*}
Let $n$ be an integer $\geq 3.$ There is no solution of the equation \[u^n +
  v^n = w^n\] with non-constant relatively prime polynomials $u,v,w$. 
\end{theorem*}



\end{frame}

\begin{frame}
  \frametitle{Proof}

  \begin{proof}
    \small
    Let $f =u^n, g = v^n,$ and $h=w^n.$ Then by the Mason-Stothers theorem
    \[
      \operatorname{deg}u^n \leq n_0(u^nv^nw^n) - 1.
    \]
    However, $\operatorname{deg}u^n=n \cdot \operatorname{deg}u$ and $n_0(u^n) =
    n_0(f) \leq \operatorname{deg}u.$ Hence 
\[
  n \cdot \operatorname{deg}u \leq \operatorname{deg}u + \operatorname{deg}v +
  \operatorname{deg}w - 1.
\]
   Similarly, we can obtain the analogous inequalities for $v$ and $w.$ \\
Adding the three inequalities yields

\[
  n(\operatorname{deg}uvw) \leq 3(\operatorname{deg}uvw) -3 < 3(\operatorname{deg}uvw).
\]
Cancelling $\operatorname{deg}uvw$ yields $ n < 3$ , thus proving the theorem.
\end{proof}
\end{frame}

\end{document}