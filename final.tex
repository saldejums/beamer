\documentclass[envcountsect]{beamer}

\usetheme{Copenhagen}
\usefonttheme{professionalfonts}
\usepackage{amsthm,amssymb,amsmath}
\usepackage{fontspec}
\usepackage{breqn}

\setbeamertemplate{theorems}[numbered]

\newtheorem*{theorem*}{Theorem}


\author[Emils Kalugins]{Presented by Emils Kalugins \\~\\
{\small Paper\footnotemark{} authored by:\\ Shashank Chorge\footnotemark{} \and Juan Vargas\footnotemark{}}}
\title{Proof of Euler's $\varphi$ (Phi) Function Formula}

\institute{Univesity of Latvia}

\begin{document}
\begin{frame}
\maketitle

\footnotetext[1]{ Available at: http://scholar.rose-hulman.edu/rhumj/vol14/iss2/6 }
\footnotetext[2]{Mumbai University}
\footnotetext[3]{UNC Charlotte}

\end{frame}

\section{Introduction}

\begin{frame}
\frametitle{Formulation}

Euler's $\varphi$ function counts the number of positive integers up to a given
integer $n$ that are relatively prime to $n$. \\~\\ 


\begin{example}
Suppose $n = 36$, then there are twelve positive integers that are coprime with 36
and lower than 36: 1, 5, 7, 11, 13, 17, 19, 23, 25, 29, 31 and 35. Which means
that $\varphi(n) = 12$. 
\end{example}

\end{frame}

\begin{frame}
\frametitle{Goal}

\begin{theorem}
For all $n \in \mathbb{N}$ we have $$ \varphi(n) = n\prod_{i=1}^m
\left(1-\frac{1}{p_i}\right) = n\left(1-\frac{1}{p_1}\right) \dots
\left(1-\frac{1}{p_m} \right), $$ 
where $n = p_1^{k_1}p_2^{k_2}\dots p_m^{k_m} $ is the prime
factorization of $n$.
\end{theorem}

\end{frame}

\begin{frame}
\frametitle{Reformulation}

\begin{definition}[Greatest common divisor]
The greatest common divisor of two nonzero integers $a$ and $b$ is the largest of
all common divisors of $a$ and $b$. We denote this integer by $gcd(a,b)$. When
$gcd(a,b) = 1$, we say that $a$ and $b$ are coprime.
\end{definition}

\vfill

Now we can restate what $\varphi(n)$ does. For every $n \in \mathbb{N}$
$\varphi(n)$ denotes the number of positive integers $r$ such that $gcd(n,r) = 1$.

\vfill
\end{frame}

\section{Consecutive intervals}

\begin{frame}
\frametitle{Lemma}

\begin{lemma}
Define $G_k = \{ r \in \mathbb{N} \, | \, 0 < r < kn \text{ and } gcd(n,r) =
1\} $. Then $|G_k| = k\varphi(n).$
\end{lemma}

\begin{proof}
Let $H_n$ be the set of numbers less than $n$ that are coprime to $n$. By
Definiton 1.2, $|H_n| = \varphi(n)$. Suppose $h \in H_n$. Then any number of the
form $kn + h$ is coprime to $n$. Since this holds for all $k \in \mathbb{N}$ and
and every $h \in H_n$,
then for any given $k$ there are exactly $\varphi(n)$ coprime numbers to $n$ in
$$ E_k = \{ r \in \mathbb{N} \, | \, kn < r < k(n+1) \text{ and } gcd(r,n) = 1\}.$$
Hence, $|G_k|$ is equal to the number of intervals $k$ times $\varphi(n)$. 
\end{proof}

\end{frame}

\begin{frame}
\frametitle{Example}
\begin{example}
Consider $n = 10$, then by the previous lemma $gcd(r,n) =\allowbreak gcd(r+10,n)$. \\~\\
$H_{10} = \{1,3,7,9\}.$ \\~\\
$E_1 = \{11,13,17,19\}$,  \\
$E_2 = \{21,23,27,29\}$, \\
\dots\\
$E_k = \{10k + 1, 10k + 3, 10k + 7, 10k + 9\}$.

\end{example}
\end{frame}

\section{Special cases}

\begin{frame}
\frametitle{First case}

\begin{lemma}
Let $p$ be a prime number and $p \mid n$, then $\varphi(pn) = p\varphi(n)$.
\end{lemma}

\begin{proof}
We first note that every number that is coprime to $pn$ is also coprime to $n$.
Since $gcd(pn,n) = n \text{ and } p \mid n$ the following result follows: $gcd(r,pn) = 1$
if and only if $gcd(r,n) = 1$ for $r \in \mathbb{N}$. \\
There are $p$ intervals, each with $\varphi(n)$ numbers relatively prime to
$n$, hence to $pn$ and therefore by Lemma 2.1, the set $G_p = \{ r \in \mathbb{N} \mid 0<r<pn \text{
and } gcd(n,r) = 1\}$ has $|G_p|=p\varphi(n)$ elements.
\end{proof}

\end{frame}

\begin{frame}
\frametitle{Second case}

\begin{lemma}
Let $p$ be a prime number and $p \nmid n$, then $\varphi(pn) = (p-1)\varphi(n)$.

\end{lemma}
\begin{proof}

By Lemma 2.1 we know that $p\varphi(n)$ is the number of coprime
numbers to $n$ that are less than $pn$. Take the set of all multiples of $p$ whose
factors are coprime to $n$. The set $\{r_1p,r_2p,\dots,r_{\varphi(n)}p\}$
contains all the elements we have overcounted because $n$ is coprime to $p$ and
$r$ by definition. Subtracting this amount from the original count, we conclude
that $\varphi(pn) =p\varphi(n) - \varphi(n) = (p-1)\varphi(n).$

\end{proof}

\end{frame}

\section{General case}

\begin{frame}
\frametitle{General case}

\begin{theorem*}
For all $n \in \mathbb{N}$ we have $$ \varphi(n) = n\prod_{i=1}^m
\left(1-\frac{1}{p_i}\right), $$ 
where $n = p_1^{k_1}p_2^{k_2}\dots p_m^{k_m} $ is the prime factorization of
$n$.
\end{theorem*}
\end{frame}

\begin{frame}
\frametitle{Proof}
\begin{proof}\renewcommand{\qedsymbol}{}
We can apply Lemma 3.2 to all of the prime factors of $n$. Thus we get the
following, 
$$ \varphi(n)=\varphi(p_1^{k_1} \dots p_m^{k_m}) =
p_1^{k_1-1} \dots p_m^{k_m-1}\varphi(p_1p_2 \dots p_m).$$
Now we apply Lemma 3.1:
$$ \varphi(n) = p_1^{k_1-1} \dots
p_m^{k_m-1}(p_1-1)(p_2-1)\dots(p_m-1). $$
We can clean this up by multiplying with $\frac{p_s}{p_s}$ for all $1 \leq s
\leq m$ (cont.).
\end{proof}

\end{frame}

\begin{frame}
\frametitle{Proof}

\begin{proof}
\footnotesize
\begin{align*}
\varphi(n) &= \left(\frac{p_1}{p_1}\right) \left(\frac{p_2}{p_2}\right)\dots 
\left(\frac{p_m}{p_m}\right) p_1^{k_1-1} \dots
p_m^{k_m-1}(p_1-1)(p_2-1)\dots(p_m-1) \\
 &= p_1^{k_1}p_2^{k_2}\dots p_m^{k_m}\left(\frac{p_1 -
 1}{p_1}\right)\left(\frac{p_2 - 1}{p_2}\right) \dots \left(\frac{p_m -
 1}{p_m}\right) \\
&= n\left(1-\frac{1}{p_1}\right) \left(1-\frac{1}{p_2}\right) \dots
\left(1-\frac{1}{p_m}\right).
\end{align*}
\end{proof}

\end{frame}

\section{}
\begin{frame}
\centering
\vfil
\Large
Thank you for your attention!
\vfil
\end{frame}


\end{document}
