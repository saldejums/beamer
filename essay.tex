\documentclass[a4paper,12pt]{article}
\usepackage[top=2cm,right=2cm,left=2cm,bottom=2cm]{geometry}
\usepackage[utf8]{inputenc}
\usepackage[T1]{fontenc}
\usepackage{newtxtext,newtxmath}
\usepackage{varwidth}
\usepackage{indentfirst}
\usepackage{setspace}

\onehalfspacing
\setlength{\parindent}{1cm}

\author{Emils Kalugins}
\title{Improving the Teaching and Learning of Mathematics}
\date{}

\begin{document}
\maketitle

Everyone knows that something is wrong with the present system of mathematics
education. Each year the average grade for the state exam is declining. Last
year it was about 35\%. Pitiful. There are many known suggestions to help remedy the sitiuation. One side says, ``we need higher standards.'' The educators say, ``we
need more resources.'' Most of the opinions are wrong. The only
people who understand what is going on are the ones most often blamed: the
students. In their opinion, ``math class is boring and useless,'' and they are right.

Mathematics is \textit{the} science. One might even consider it the antitheses
of art. Right? Well, not really. The first thing to understand is that
mathematics is an art. The only difference between math and the other arts, such
as writing and painting, is that our culture does not recognize it as
such. Everyone understands that musicians produce works of art, and are
expressing themselves in sound. So what is the barrier that keeps the society
from recognizing mathematics as working artists? Part of it is ignorance. Nobody
has the slightest idea what it is that mathematicians do. But that is not even
the worst part. What is far worse is that they think they do know what math is
about. They see maths as a tool for scientists and technology, that it is
somehow useful to society. So if mathematicians do not just multiply large numbers
or calculate the areas of triangles, then what do they do? Who else is to
describe what a mathematician is than a \textit{real} mathematician? In the
words of G. H. Hardy:

\begin{center}

\begin{varwidth}{0.5\textwidth}
A mathematician, like a painter or poet, is a maker of patterns. If his patterns
are more permanent then theirs, it is because they are made with
\textit{ideas}.

\raggedleft
(A Mathematician's apology)
\end{varwidth}

\end{center}

The next question that logically arises is about the nature of the ideas and
patterns involved. Are they something grand concerning culture? Mostly not. The
governing principle in mathematics is that \textit{simplicty is beautiful}. The
simplest things are imaginary. No real world square is perfect. There will
always be limitations on the precision we can measure something, even without
considering the fact that real objects are everchanging as their atoms move. The
mathematicians square is perfect because he wants it that way. That is the whole
aesthetic of mathematics; there are no ugly considerations of real-world
details. So mathematicians get to imagine whatever they want and ask questions
about them. How does one answer these questions? After all the object of interest
does not even exist! Here is where it differs from science. There are no
experiments to be run to tell the truth about a fictious object in our
imagination. The only way to get the truth is to use our imaginations, and that
is hard work.

``The main problem with school mathematics is that there are no
\textit{problems},'' is the way that Paul Lockhart describes maths education in
his essay (now expanded to a book) \textit{A Mathematician's Lament}. A teacher
gives their students a formula to memorize and asks them to ``apply'' it over
and over in the exercises. The whole thrill is gone. There is no discovery, even
the frustration that one feels in the creative act. So to supplement this lack
of excitement, the teachers try to make math interesting. But that is not
necessary! Math is more interesting than anyone can handle. The whole reason it
is fun is because of its irrelevance to our lives. There is nothing wrong with
facts, but that should not be the main goal. What matters are the ideas behind
these facts. As discussed earlier, they do not appear from thin air; they were
discovered. To drive this point home, Lockhart again gives a great analogy, ``It
is like \textit{saying} that Michelangelo created a beautiful sculpture, without
letting me \textit{see} it. How am I supposed to be inspired by that?'' But the
situation is actually worse. Here it is understood that there \textit{is} a
sculpture to be appreciated. When only facts are given the complete story is
omitted and the existance of the beauty or idea behind the mathematical fact is left
uncertain. Mathematics, in a way, could also be categorized as the art of
explanation.

The correct problem to give a student is one which he does not know
how to solve. Of course it also has to be within his grasp for the problem to be
fruitful. This is where a teacher steps in. He is the one who can evaluate the
difficulty of a problem. The teacher can guide the student in the right
direcion, if the situation calls for it, but it should never be rushed! Art, and
in turn mathematics, is not a race. The problem arises from the fact that
mathematics is hard creative work. It is a slow, contemplative proces. To really
practice mathematics the student should struggle. The teacher should step in
only when the student is desperate.

The sad truth is that all of this is virtually impossible. Even if it were, not
many would want such a personal connection with their students. Not to mention
the ammount of work that needs to be put in such a responsible task. It is a lot
easier to be a passive tape-recorder that just follows a given set of
instructions from a standard textbook. What this achieves is just a trained
monkey that can reiterate the solution to an already solved problem. It's
a sad way to learn mathematics. As Lockhart puts it, ``It is simply the 
path of least resistance.''

The current education reform is proof that something is rotting. The key here is
 to finally give up on trying to save the Titanic. It is a lost cause. By making
the curriculum ``relatable'' and ``important'', the whole spirit of
mathematics is killed. One must accept the fact that mathematics is a form of
art and should be treated as such. Since Bablyonian times math has existed
purely for pleasure. So the resolution to improve maths education is to actually
teach mathematics.
\end{document}
