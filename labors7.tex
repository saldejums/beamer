\documentclass[a4paper,12pt]{article}
\usepackage[top=2cm,bottom=2cm,right=2cm,left=2cm]{geometry}
\usepackage{polyglossia}
\usepackage{floatrow}
\usepackage{graphicx}
\usepackage{amsmath,amssymb}
\setdefaultlanguage{latvian}
\usepackage{tabu}
\usepackage{array}
\usepackage{adjustbox}
\usepackage{soul}
\usepackage{xunicode}
\usepackage{fixlatvian}
\usepackage{paracol}
\usepackage{indentfirst}
\setlength{\parindent}{1cm}
\setlength{\parskip}{3pt}
\usepackage[document]{ragged2e}
\floatsetup[table]{capposition=top}

\begin{document}

\RaggedLeft
  Latvijas Universitāte \\
  Fizikas un matemātikas fakultātes \\
  Matemātikas BSP 2. kursa students \\ 
  Emīls Kalugins (ek16073), Lauma Luckāne un Mārtiņš Gailītis\\ 
  11.04.2018 \\

\Centering
{\large{Darbs Nr. 47}} \\ ~ \\ 

\textit{\large{Šķidruma viskozitātes noteikšana ar Stoksa metodi}} \\ ~ \\ 

\RaggedRight
\columnratio{0.25}

\begin{paracol}{2}
  \ul{Darba mērķis:}
  \switchcolumn
  Iepazīties ar šķidrumu viskozitātes noteikšanas metodi. Novērtēt viskozitātes
  lomu dabā un tehnikā.
   \switchcolumn*
  \ul{Darba uzdevumi:}
  \switchcolumn
  \textbf{1. Pārbaudīt atsevišķas lodītes grimšanas.
    \\
    2. Noteikt glicerīna viskozitātes koeficientu ar Stoksa metodi.\\
  3. Salīdzināt iegūto viskozitātes koeficientu ar darba aprakstā pieejamiem
  datiem par glicerīna viskozitāti pie dažādām temperatūrām.\\}

  \switchcolumn*
  \ul{Darba piederumi:}

  \switchcolumn
  Mērmikroskops (Apjoms - 3.00 mm, Kļūda - 0.02 mm)\\
  Cilindrs ar glicerīnu (Diametrs - 3.2 cm, Kļūda - 0.2 cm)\\
  Hronometrs (Sistemātiskā kļūda - 0.09 s)

\end{paracol}

\justify

\noindent
\ul{Jautājumi:}\\

\noindent
1. Kas ir viskozitāte?\\
Tā ir iekšējas berzes spēks starp šķidruma vai gāzes slāņiem.\\
2. Uzrakstiet Ņūtona 2. likumu un atšīfrējiet. \\
F = ma. m - masa kg, F - spēks, N, a - paātrinājums, $\text{m/s}^2$.\\
3. Pieņemsim, ka laboratorijā atrodas kaste, kurā ir vakuums. Jūs kastes
augšpusē ievietojat mazu metāla lodīti un palaižat to vaļā. Kas notiek ar
lodīti? Kā mainās tās ātrums? Kā mainās tās paātrinājums?\\ 
Uz to darbosies tikai smaguma spēks, un tā kā tas ir vakuums, tad nav berzes un
tā paātrināsies neierobežoti ar paātrinājumu, kas vienāds ar brīvās krišanas
paātrinājumam, līdz nokritīs. \\
4. Jūs ieliekat lodīti traukā ar glicerīnu (augšpusē) un palaižat to vaļā. Kādi
3 spēki iedarbojas uz lodīti? Kādos virzienos ir vērsti šie spēki? \\
Darbosies smaguma spēks, Arhimēda spēks un šķidruma pretestības spēks. Pirmais spēks būs vērsts
uz ``leju'', bet pārējie uz ``augšu''.\\
5. Kuri no iepriekšējā punktā aprakstītajiem spēkiem ir atkarīgi no lodītes
ātruma? Kuri no spēkiem ir atkarīgi no viskozitātes koeficienta? \\
Abi šie spēki ir šķidruma pretestības spēks.\\
6. Ja lodīte kustas ar konstantu ātrumu, ko ir iespējams pateikt par spēkiem,
kas darbojas uz lodīti? \\
To kopējā summa ir 0 jeb pēc moduļa pretēji vērstie spēki ir vienādi.\\







\noindent
\ul{Teorētiskais pamatojums:}\\

Stokss atklāja, ka, ja lodīte ar rādiusu $r$ kustas šķidrumā ar nelielu
ātrumu $v$, tad šķidruma pretestības spēks $F$ izsekāms šādi -
$$
F = 6\pi\eta rv,
$$
kur $\eta$ - šķidruma viskozitātes koeficients. Ja lodīti palaiž šķidrumā, tad
smaguma spēka dēļ tā paātrināsies līdz brīdim, kad pretestības spēks kopā ar
Arhimēda spēku izslēgs smaguma spēku un lodīte kustēsies ar konstantu ātrumu. Ja
ar $\rho$ apzīmē lodītes blīvumu un $\rho_0$ šķidruma blīvumu, tad varam
secināt, ka pie konstanta ātruma iestāšanas brīža spēkā ir sekojošais:
$$
\frac{4}{3}\pi r^3 g = 6\pi\eta rv +\frac{4}{3}\pi r^3 \rho_0 g.
$$
Tad, lai atrastu viskozitātes koeficientu, izsaka $\eta$ un iegūst -
$$
\eta = \frac{1}{18}\cdot g\cdot (\rho-\rho_0)\cdot \frac{d^2}{v_{max}}.
$$

Bet, ja ir konstants ātrums, tad to var izteikt kā attiecību starp ceļu $s$,
kurā lodīte noceļo laikā $t$. Vēl papildus var ņemt vērā to, ka Stokss savu
likumu veidojis bezgalīgam tilpumam, kur neņem vērā sienas iedarbību, tāpēc
ātrums tiek koriģēts ar Ladenburga korekciju -
$$
v_{max} = v_{noteiktais}\cdot\left(1+2.4\frac{d}{D}\right),
$$
kur $D$ - diametrs tvertnei. Rezumējot iegūstam sakarību, ko arī izmanto
eksperimentā, -
$$
\eta = \frac{1}{18}\cdot g\cdot (\rho-\rho_0)\cdot \frac{d^2\cdot t}{s\cdot \left(1+2.4\frac{d}{D}\right)}.
$$
\newpage

\noindent
\ul{Mērskaitļu tabula:} \\

\begin{table}[h!]
  \begin{adjustbox}{width=1\textwidth}
  \begin{tabular}{||c|c|c|c|c|c|c|c|c|c||}
  \hline 

   \hline 
  \end{tabular}
  \end{adjustbox}

  \caption{Materiāla liece}
\end{table}

\noindent
\ul{Aprēķinu piemēri:}\\

\noindent


\noindent
\newpage

\noindent
\ul{Grafiks:}

\begin{figure}[h!]
  \centering
\includegraphics[width = 0.9\textwidth]{visc}
\caption{Glicerīna viskozitātes koeficienta atkarība no temperatūras}
\end{figure}


\noindent
\ul{Secinājumi:} \\


\end{document}
